

% Coordenador(a) atual do Curso
\coursecoord{Alceu Heinke Frigeri}[m]
\coursecoordtitle{Coordenador de Curso}

% o nome do curso pode ser redefinido (ex. para Monografias)
%
%% \courseacronym{CCA}
%% \course{Eng. de Controle e Automação}

%\universitydivision{Escola de Engenharia}
%\university{Universidade Federal...}

%% for a report
%
%\courseundef
%\department{DELAE - Dept. de Sistemas Elétricos e Energia}
%\class{ENG100xy}{some thing else}
%\subject{Segundo trabalho da disciplina}

% o local de realização do trabalho pode ser especificado (ex. para Monografias)
% com o comando \location:
%\location{São José dos Campos}{SP}


%%%%%%%%%%%%%%%%%%%%%%%%%%%%%%%%%%%%%%
%%%%%%%%%%%%%%%%%%%%%%%%%%%%%%%%%%%%%%
%%%%%%%%%%%%%%%%%%%%%%%%%%%%%%%%%%%%%%


% Informações gerais
%
\title{Mapeamento de ambiente para navegação de um robô móvel}

\author{Scharlau Coelho}{Henrique}
\authorinfo{243627}{henriquescharlaucoelho@gmail.com}
% alguns documentos podem ter varios autores:
%\author{Flaumann}{Frida Gutenberg}
%\author{Flaumann}{Klaus Gutenberg}

% orientador
\advisor[Prof.~Dr.]{Fetter Lages}{Walter}[m]
\advisorinfo{UFRGS}{Doutor pelo Instituto Tecnológico de Aeronáutica -- São José dos Campos, Brasil}{fetter@ece.ufrgs.br}{ramal}

% O comando \advisorwidth pode ser usado para ajustar o tamanho do campo
% destinado ao nome do orientador, de forma a evitar que ocupe mais de uma linha 
\advisorwidth{0.55\textwidth} % só funcionou para a página 2

% obviamente, o co-orientador é opcional
%\coadvisor[Prof.~Dr.]{do Co-orientador (se houver)}{Nome}
%\coadvisorinfo{UFRGS}{Doutor pela (Instituição onde obteve o título -- Cidade, País)}{email}{ramal}
%\coadvisorgender %% the default is [m], other option is [f]

% banca examinadora
\examiner[Prof.~Dr.]{Ventura Bayan Henriques}{Renato}[m]
\examinerinfo{UFRGS}{Doutor pela Universidade Federal de Minas Gerais -- Belo Horizonte, Brasil}{rventura@ece.ufrgs.br}{}

\examiner[Prof.~Dr.]{Antônio Comparsi Laranja}{Rafael}[f]
\examinerinfo{UFRGS}{Doutor pela Universidade Federal do Rio Grande do Sul -- Porto Alegre, Brasil}{rafael.laranja@ufrgs.br}{3749}


% suplentes da banca examinadora (apenas para alguns formulários)
\altexaminer[Prof.~Dr.]{do professor suplente I)}{(nome}
\altexaminerinfo{sigla da Instituição I onde atua}{Doutor pela (Instituição Ia onde obteve o título -- Cidade, País)}{email}{ramal}


%% resumo do trabalho (para o formulário de renovação de requerimento de matrícula.
%%
\tccbrief{INSERIR RESUMO AQUI}
\tccadvisorsreview{}
\tcccoadvisorbrief{justificativa para ter-se um co-orientador...}

% palavras-chave
% iniciar todas com letras minúsculas, exceto no caso de abreviaturas
%
\keyword{Robótica}
\keyword{Navegação autonôma}
\keyword{Mapeamento de ambientes}

% a data deve ser a da defesa; se nao especificada, são gerados
% mes e ano correntes
%\date{fevereiro}{2004}


