% Meta-monografia de exemplo genérico de uso da classe delaetex.cls
% Copyright (C) 2004..2016 Walter Fetter Lages <fetter@ece.ufrgs.br>
%
% This file was adapted from:
% Meta-monografia de exemplo genérico de uso da classe deletex.cls
% Copyright (C) 2004 Walter Fetter Lages <w.fetter@ieee.org>
%
% This is free software, distributed under the GNU GPL; please take
% a look in `deletex.cls' to see complete information on using, copying
% and redistributing these files
%
%\documentclass[repeatfields,openright,overleaf,nomicrotype]{tcc}
\documentclass[repeatfields,xlists,xpacks,oneside,yearsonly]{ufrgscca}
\addbibresource{modeloTCC.bib}

%
% inicio do documento
%
%\RenewDocumentCommand{\fullcite}{}{}
%\RenewDocumentCommand{\cite}{}{}
%\let\MakeUppercase\uppercase
%\RenewDocumentCommand{\MakeUppercase}{m}{\uppercase{#1}}

\begin{document}

\maketitle


% dedicatoria é opcional
%\nonum\chapter{Dedicatória} %vai ter uma entrada no sumário
%\notoc\chapter{Dedicatória} %não vai aparecer no sumário

% Dedico este trabalho aos meus pais, em especial pela dedicação e apoio em
% todos os momentos difíceis.

% agradecimentos são opcionais
%\nonum\chapter{Agradecimentos}
%\notoc\chapter{Agradecimentos}

% \`{A} Universidade Federal do Rio Grande do Sul, UFRGS, pela
% oportunidade de realização de estudos.

% Aos colegas de curso pelo seu auxílio nas tarefas desenvolvidas durante o
% curso e apoio na revisão deste trabalho.

% Agradeço ao \LaTeX\ por não ter vírus de macro\ldots

% resumo no idioma do documento
\begin{abstract}

\end{abstract}

% resumo no outro idioma
% como parametro devem ser passadas as palavras-chave
% no outro idioma, separadas por vírgulas
% \begin{otherabstract}{Electrical Engineering, Signal Processing, Automation and Control, Electronic and Instrumentation}

% \end{otherabstract}

% sumario
\setcounter{tocdepth}{3}

% lista de ilustrações
\listoffigures

% lista de tabelas
\listoftables

% lista de listagens (código fonte)
\listofcodelist %% doesn't work on overleaf

% lista de abreviaturas e siglas
% o parametro deve ser a abreviatura mais longa
\begin{listofabbrv}{PPGEE}
    \item[ROS] Robot Operating System
\end{listofabbrv}

% lista de símbolos é opcional
\begin{listofsymbols}{$\alpha\beta\pi\omega$}
    \item[$\sum$] Somatório
\end{listofsymbols}

\tableofcontents


\chapter{Introdução}

\chapter{Revisão da Literatura}
\label{revisao}

\chapter{Metodologia}
\label{partes}

\chapter{Conclusão}
\label{conclusao}

\printbibliography

\begin{appendix}

\end{appendix}


\begin{annex}
\end{annex}

\end{document}

